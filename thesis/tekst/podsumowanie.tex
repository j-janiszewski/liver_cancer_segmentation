This work aimed to review and summarize the current state of knowledge regarding detecting liver cancer tissue from medical images and develop a program to perform such detection. The findings of this work show that methods based on deep learning achieve very high efficiency in segmenting liver cancer lesions; at the same time, this research area is very active, and new competing solutions in terms of efficiency regularly appear. The implications for the medical industry could be increased availability of liver cancer diagnostics in the near future. Another finding is that there is no established standard for pre-processing medical images, with researchers using different methodologies or parameters. In developing the program above, the best-performing Hounsfield windowing parameters were values not used in any of the studies.
The last main finding is that nowadays, there are open-source libraries that allow high-level components to be used to create segmentation models with architectures corresponding to the current state of knowledge. This makes the study of this issue available to those with advanced programming skills and allows research to be more interdisciplinary.

The practical implication of this work is that it is possible to develop a solution that can be used in clinical practice inspired by the one described. Current studies often describe methodologies and research while not sharing the source code and not describing it in detail, making it difficult to reproduce the results. As for the theoretical implications, research gaps, such as comparing different pre-processing methods with each other on the same data set, have been identified in addition to the value of summarizing current knowledge.

As for the limitations on this work, it should be mentioned that the dynamics in which new discoveries in this area are occurring may cause this summary to be soon outdated. However, as a point of reference, this work will continue to have significant value. Experiments conducted on the developed program could include more patients, taking into account a larger sample of those who also have healthy livers. This is a problem faced by researchers around the world, due to legal issues and the workload involved in developing the data, we are currently limited to a small amount of data to verify the performance of the methods.