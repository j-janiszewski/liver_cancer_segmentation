Liver cancer is a significant health concern, in 2020, it was the second most common cause of cancer-related death worldwide. Despite being one of the most common causes of death it is not the most common cancer, liver cancer accounted for $8.3\%$ of all deaths but only $4.7\%$ of all detected cases in 2020 \cite{sung_global_2021}. The reason for this increased mortality is the difficulty of diagnosing it at an early stage due to the lack of symptoms. Being a complex disease, the development and growth of liver cancer involve multiple microscopic and macroscopic changes in cell morphology, which are not yet fully understood. Chances of diagnosis at an early stage are provided by identification of these lesions on medical images of the liver. Some of the most common techniques for performing patient imaging include:
\begin{itemize}
    \item Ultrasound (US)
    \item Magnetic resonance imaging (MRI)
    \item Computed tomography (CT)
\end{itemize}

Each of these techniques has its own advantages and disadvantages related to aspects such as imaging cost, image quality and invasiveness to the patient \cite{li_application_2020}. The irregular shapes of the liver and tumor lesions make segmentation based on medical images a non-trivial task. Over the years, many techniques have emerged to ease this process. However, none of the classical methods allowed for the preservation of high accuracy of segmentation and for making it applicable to medical practices in high-volume settings. The need for high accuracy is obvious, while the possible scalability of the solution would allow for an increase in the detection of liver cancer among the entire population, and not only the wealthiest part would be able to pay for the time of specialized experts.


 In recent years, deep learning methods have revolutionized image processing techniques. Models developed based on these methods can automatically extract hierarchical representations of image features from raw input data. They can recognize complex patterns and representations found in images at different levels of abstraction \cite{anguera_impact_2018}. Given the level of complexity and the multidimensionality of medical images, methods based on deep learning are considered highly promising and likely to revolutionize medical radiology in the near future.

Accordingly, this work aims to summarize the current research and state of the art in detecting malignant lesions on medical images, focusing on methods using deep learning while mentioning classical ones. Apart from it the goal is to create a program that performs the automatic segmentation of CT images using deep learning and evaluate its performance.


